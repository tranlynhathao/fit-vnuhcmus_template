\subsection{Programming note}
Chrono library: use for measuring running time of sorting functions
\begin{itemize}[label=-]
    \item \texttt{<chrono>} is a \texttt{C++} header that provides a collection of types and functions to work with time. It is a part of the \texttt{C++} Standard Template Library (STL) and it’s included in \texttt{C++11} and later versions. ~\cite{ref13}
    \item \texttt{<chrono>} provides three main types of clocks: 
        \begin{itemize}[label=$\bullet$]
            \item \texttt{system\_clock}, \texttt{steady\_clock}, and \texttt{high\_resolution\_clock}. 
            
            \item These clocks are used to measure time in various ways. ~\cite{ref13}
        \end{itemize}
    \item Here is a pseudocode:
        \begin{itemize}[label=$\bullet$]
            \item \texttt{start\_time = get\_current\_high\_resolution\_time()} (record the current time)
            
            \item \texttt{runSortingAlgorithmWithTime(a, n, algoIndex)} (calling function)
            
            \item \texttt{end\_time = get\_current\_high\_resolution\_time()} (record the current time)
            
            \item \texttt{duration = end\_time - start\_time} (calculate the duration in milliseconds)
            
            \item \texttt{time = convert\_to\_milliseconds(duration)} (convert duration to milliseconds)
        \end{itemize}
\end{itemize}
