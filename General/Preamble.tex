% Essential for making this template work are graphicx, float, tabularx, tabu, tocbibind, titlesec, fancyhdr, xcolor and tikz. 

% Not essential, but you will have to debug the document a little bit when removing them are amsmath, amsthm, amssymb, amsfonts, caption, subcaption, appendix, enumitem, hyperref and cleveref.

% inputenc, lipsum, booktabs, geometry and microtype are not required, but nice to have.

% Essential packages
\usepackage[utf8]{inputenc} % Allows the use of some special characters
\usepackage{amsmath, amsthm, amssymb, amsfonts} % Nicer mathematical typesetting
\usepackage{lipsum} % Creates dummy text lorem ipsum to showcase typsetting 
\usepackage{graphicx} % Allows the use of \begin{figure} and \includegraphics
\usepackage{float} % Useful for specifying the location of a figure ([H] for ex.)
\usepackage{caption} % Adds additional customization for (figure) captions
\usepackage{subcaption} % Needed to create sub-figures
\usepackage{tabularx, tabu, booktabs, multirow} % Adds additional customization for tables
\usepackage[nottoc,numbib]{tocbibind} % Automatically adds bibliography to ToC
\usepackage[margin=2.5cm]{geometry} % Allows for custom (wider) margins
\usepackage{titletoc} % Used to create a custom ToC
\usepackage{appendix} % Any chapter after \appendix is given a letter as index
\usepackage{fancyhdr} % Adds customization for headers and footers
\usepackage[shortlabels]{enumitem} % Adds additional customization for itemize 
\usepackage{hyperref} % Allows links and makes references and the ToC clickable
\usepackage[noabbrev, capitalise]{cleveref} % Easier referencing using \cref{<label>} instead of \ref{}
\usepackage{xcolor} % Predefines additional colors and allows user defined colors
\usepackage{tikz} % Useful for drawing images, used for creating the frontpage
\usetikzlibrary{positioning, calc, shapes.geometric, arrows, arrows.meta} % Additional libraries for tikz
\usepackage{chngcntr}
\usepackage{xspace}
\usepackage{longtable}
\usepackage{multirow}
\usepackage{sectsty}
\subsubsectionfont{\normalfont\normalsize\bfseries}
\usepackage{lscape}
\usepackage{listings}
\usepackage{microtype}
\usepackage{tocloft}
\usepackage{url}
\usepackage{algorithm}
\usepackage{algpseudocode}
\usepackage{ifxetex}
\usepackage{ifluatex}
\usepackage{vntex}
\usepackage[english]{babel}
\usepackage{fontawesome}
\usepackage{wasysym}
\usepackage{titlesec}
\usepackage{natbib}
% \usepackage{algorithmic}

% Defines a command used by tikz to calculate some coordinates for the front-page
\makeatletter
\newcommand{\gettikzxy}[3]{%
  \tikz@scan@one@point\pgfutil@firstofone#1\relax
  \edef#2{\the\pgf@x}%
  \edef#3{\the\pgf@y}%
}
\makeatother
